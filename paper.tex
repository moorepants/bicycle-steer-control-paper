\documentclass[12pt]{article}

\usepackage{amsmath}
\usepackage{graphicx}

\title{Summative Robotic Bicycle Steer Control}

\author{Jason K. Moore}
\date{\today}

\begin{document}
\maketitle

\section{Controllability}

\begin{figure}
  \centering
  \includegraphics{figures/uncontrolled-with-rider-geometry.png}
  \caption{geometry}
  \label{fig:uncontrolled-with-rider-geometry}
\end{figure}


The linear Whipple-Carvallo model can be described by the state space
equations:

\begin{align}
  \dot{\bar{x}} = \mathbf{A} \bar{x} + \mathbf{B} \bar{u} \\
  \bar{x} = \begin{bmatrix} \phi \\ \dot{\phi} \\ \delta \\ \dot{\delta}
  \end{bmatrix} \\
  \bar{u} = \begin{bmatrix} T_{\phi} \\ T_{\delta} \end{bmatrix}
\end{align}

The controllability matrix \(\mathbf{R}\) can be formed for the steer input.
There are speed parameter values where the linear Whipple-Carvallo bicycle
model is uncontrollable. Solving the \(det(\mathbf{R}(v))=0\) returns two
speeds in the range \([0, 10]\) m/s that are uncontrollable.

\begin{figure}
  \centering
  \includegraphics{figures/uncontrolled-eigenvalues-with-rider.png}
  \caption{Real and imaginary parts of the eigenvalues as a function of speed.
  Vertical black lines indicate uncontrollable dynamics.}
  \label{fig:uncontrolled-eigenvalues-with-rider}
\end{figure}

I have no idea why this model is uncontrollable at these speeds.

We can also look at the modal controllability to see the degree of
controllability Fig.~\ref{fig:modal-controllability}.

\begin{figure}
  \centering
  \includegraphics{figures/modal-controllability.png}
  \caption{Testing}
  \label{fig:modal-controllability}
\end{figure}

I don't think this model controllability calc is correct because it isn't
smooth and it doesn't reveal the two uncontrollable speeds.

\section{Stabilization with Roll and Steer Feedback}

One IMU on the handlebars and one on the rear frame easily gives: roll rate and
steer rate. Steer angle sensor gives steer angle. Roll angle must be estimated.

Single motor that applies a torque between the rear frame and the
handlebars/fork.

A full state feedback takes this form:

\begin{align}
  \dot{\bar{x}} = \mathbf{A} \bar{x} + \mathbf{B} \bar{u} \\
  \bar{x} = \begin{bmatrix} \phi \\ \dot{\phi} \\ \delta \\ \dot{\delta} \end{bmatrix} \\
  \bar{u} = \left[ T_\delta \right] \\
  T_\delta = T_{\delta,\textrm{human}} + T_{\delta,\textrm{robot}} \\
  T_{\delta,\textrm{robot}} = k_\phi \phi + k_{\dot{\phi}} \dot{\phi} \\
  \dot{\bar{x}} = \left( \mathbf{A} - \mathbf{B} \mathbf{K} \right) \bar{x} + \mathbf{B} \left[ T_{\delta,\textrm{human}} \right] \\
  \mathbf{K} = \begin{bmatrix} k_\phi & k_{\dot{\phi}} & 0 & 0 \end{bmatrix}
\end{align}

Note that the rider still applies steer torque through the same B matrix and
any roll disturbances also through same B matrix.

\section{Proportional Dervivative Roll Control}

Proportional Derivative Roll Control

\begin{itemize}
  \item What are the achievable closed loop dynamics?
  \item Vary speed and controller gain, show 3D root locus
  \item Pick a speed and vary the controller gain, show 2D root locus
  \item Difference with and without a rigid rider
  \item Could be PD on roll rate criteria (overshoot, decay time) and then pole
    placement with the two parameters for each speed to get similar behavior
    across speeds.
  \item What if we only have rate feedback (raw IMU data)? Does steer rate
    feedback help?
\end{itemize}

It is well known that at low speeds simple proportional positive feedback of
roll rate stabilize the bicycle.

\begin{figure}
  \centering
  \includegraphics{figures/roll-rate-eig-effect.png}
  \caption{Testing}
  \label{fig:roll-rate-eig-effect}
\end{figure}

Roll rate feedbac is not sufficient to stabilize the bicycle at very low speeds
(< 0.5 m/s) and speeds where the capsize mode is unstable. But if steer rate
and roll rate are avaiable to feedback, the bicycle can be stabilized both at
higher speeds. Exponential functions provide a good model for gain scheduling
with respect to speed.

\begin{figure}
  \centering
  \includegraphics{figures/pd-gains-vs-speed.png}
  \caption{Testing}
  \label{fig:pd-gains-vs-speed}
\end{figure}

\begin{figure}
  \centering
  \includegraphics{figures/pd-eigenvalues.png}
  \caption{Testing}
  \label{fig:pd-eigenvalues}
\end{figure}

\begin{figure}
  \centering
  \includegraphics{figures/pd-simulation.png}
  \caption{Testing}
  \label{fig:pd_simulation}
\end{figure}

\section{Full State Feedback}

- I could frame the whole controller as a full state feedback of essential
bicycle state variables and then show degenerate cases. So [kphi kphidot kdelta
kdeltadot]
- Look at LQR solution for full state feedback and limit the rider steer torque
- What are the achievable closed loop dynamics? (8 gains now)
- Not sufficient to balance at zero speed (what is lowest speed?)
- Show that speed can't affect the rank of the controlability matrix
(analytically)

If you assume full state feedback and we drive the state to zero, an LQR
controller can be realized at each speed. Assumed Q and R to be identity.

\begin{figure}
  \centering
  \includegraphics{figures/lqr-eig.png}
  \caption{Testing}
  \label{fig:lqr-eig}
\end{figure}

\begin{figure}
  \centering
  \includegraphics{figures/lqr-gains.png}
  \caption{Testing}
  \label{fig:lqr-gains}
\end{figure}

\begin{figure}
  \centering
  \includegraphics{figures/lqr-simulation.png}
  \caption{Testing}
  \label{fig:lqr-simulation}
\end{figure}

\section{What steer torques are required for normal manuvers?}

\section{Effect of Electric Motor Dynamics}

- Limitations in motor power and bandwidth

\section{Can we mimic other real bicycles?}

- Optimization that finds controller parameters for matching dynamics?

\section{Comparison of the gyrobike and the steer motor}

- Show how the same dynamics can be produced
- Energy costs comparison

\section{How does this effect thigns with a leaning rider?}

\section{Add human control}

- Handling quality metric

\section{Is the stability deteriminal to maunerability?}

\end{document}
