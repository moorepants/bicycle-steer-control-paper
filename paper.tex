\documentclass[12pt]{article}

\usepackage{amsmath}
\usepackage{graphicx}

\title{Summative Robotic Bicycle Steer Control}

\author{Jason K. Moore}
\date{\today}

\begin{document}
\maketitle

\section{Introduction}

\cite{Ruijs1986a} demonstrates that roll angle feedback stabilizes the capsize
mode and roll rate feedback stabilizes the weave mode of a motorcycle.

The purpose of the paper is to try to answer this question:

Given a bicycle that has a motor (with power, torque, speed, bandwidth limits)
which can apply a torque between the rear and front frames and a set of sensors
that can give good measurements or estimates of the bicycle's steer angle,
steering rate, roll angle, and rolling rate what closed loop dynamics and
motion are possible and how might a rider controlling this bicycle in a
summative fashion find the behavior?

Some other notes:

\begin{itemize}
  \item we only care about the lateral dynamics and handling qualities of
  \item we will only explore mathematical and computational models
\end{itemize}

\section{Model}

The linear Whipple-Carvallo model can be described by the state space
equations:

\begin{align}
  \dot{\bar{x}} = \mathbf{A} \bar{x} + \mathbf{B} \bar{u} \\
  \bar{x} = \begin{bmatrix} \phi \\ \dot{\phi} \\ \delta \\ \dot{\delta}
  \end{bmatrix} \\
  \bar{u} = \begin{bmatrix} T_{\phi} \\ T_{\delta} \end{bmatrix}
\end{align}

\begin{figure}
  \centering
  \includegraphics[width=\columnwidth]{figures/uncontrolled-with-rider-geometry.png}
  \caption{Geometry, mass, and inertia of the bicycle-rider system.}
  \label{fig:uncontrolled-with-rider-geometry}
\end{figure}

\section{Controllability}

The controllability matrix \(\mathbf{C}\) can be formed for the steer input,
\(T_\delta\). There are speed parameter values where the linear
Whipple-Carvallo bicycle model is uncontrollable. Solving the
\(det(\mathbf{C}(v))=0\) returns two speeds in the range \([0, 10]\) m/s that
are uncontrollable.

\begin{figure}
  \centering
  \includegraphics[width=\columnwidth]{figures/uncontrolled-eigenvalues-with-rider.png}
  \caption{Real and imaginary parts of the eigenvalues as a function of speed.
  Vertical black lines indicate uncontrollable dynamics.}
  \label{fig:uncontrolled-eigenvalues-with-rider}
\end{figure}

I have no idea why this model is uncontrollable at these specific speeds.

We can also look at the modal controllability to see the degree of
controllability Fig.~\ref{fig:modal-controllability}.

\begin{figure}
  \centering
  \includegraphics[width=\columnwidth]{figures/modal-controllability.png}
  \caption{Testing}
  \label{fig:modal-controllability}
\end{figure}

I don't think this model controllability calc is correct because it isn't
smooth and it doesn't reveal the two uncontrollable speeds.

\section{Stabilization with Roll and Steer Feedback}

One IMU on the handlebars and one on the rear frame easily gives: roll rate and
steer rate. Steer angle sensor gives steer angle. Roll angle must be estimated.

Single motor that applies a torque between the rear frame and the
handlebars/fork.

Assume that the torque between the two frames is the sum of the motor and the
rider:

\begin{align}
  T_\delta = T_{\delta,\textrm{human}} + T_{\delta,\textrm{motor}}
\end{align}

The motor is driven by a full state feedback controller:

\begin{align}
  T_{\delta,\textrm{motor}} =
  k_\phi \phi +
  k_\delta \delta +
  k_{\dot{\phi}} \dot{\phi} +
  k_{\dot{\delta}} \dot{\delta}
\end{align}

The motor closed loop dynamics is then:

\begin{align}
  \dot{\bar{x}} = \left( \mathbf{A} - \mathbf{B} \mathbf{K} \right) \bar{x} +
  \mathbf{B} \left[ T_\phi \quad T_{\delta,\textrm{human}} \right]^T \\
\end{align}

where

\begin{align}
  \mathbf{K} =
  \begin{bmatrix}
    0 & 0 & 0 & 0 \\
    k_\phi & k_\delta & k_{\dot{\phi}} & k_{\dot{\delta}}
  \end{bmatrix}
\end{align}

Note that the rider still applies steer torque through the original B matrix
and any roll disturbances also through same B matrix. The human controlled
dynamics can be manipulated by changing the motor controller gains.

\section{Proportional Dervivative Roll Control}

Proportional Derivative Roll Control

\begin{itemize}
  \item What are the achievable closed loop dynamics?
  \item Vary speed and controller gain, show 3D root locus
  \item Pick a speed and vary the controller gain, show 2D root locus
  \item Difference with and without a rigid rider
  \item Could be PD on roll rate criteria (overshoot, decay time) and then pole
    placement with the two parameters for each speed to get similar behavior
    across speeds.
  \item What if we only have rate feedback (raw IMU data)? Does steer rate
    feedback help?
\end{itemize}

It is well known that at low speeds simple proportional positive feedback of
roll rate stabilize the bicycle. This is positive roll derivative feedback.

\begin{figure}
  \centering
  \includegraphics[width=\columnwidth]{figures/roll-rate-eig-effect.png}
  \caption{Effect on the motor controlled closed loop dynamics with changing
  \(k_{\dot{\phi}}\). Blue to yellow varies the gain from zero to a large
  value.}
  \label{fig:roll-rate-eig-effect}
\end{figure}

Roll rate feedback alone is not sufficient to stabilize the bicycle at very low
speeds (< 0.5 m/s), even with infinite gains, and speeds where the capsize mode
is unstable. But if roll angle and roll rate are available to feedback, the
bicycle can also be stabilized at the higher speeds. If you gain schedule roll
angle and roll rate feedback to both stabilize sufficiently while minimizing
the weave frequency exponential functions provide a good model for gain
scheduling with respect to speed.

\begin{figure}
  \centering
  \includegraphics[width=\columnwidth]{figures/pd-gains-vs-speed.png}
  \caption{Exponentially scheduled roll angle and roll rate gains with respect
  to speed.}
  \label{fig:pd-gains-vs-speed}
\end{figure}

The roll PD gain scheduling with respect to speed stabilizes at all speeds
above about 0.75 m/s and retains similar dynamics except that the weave
frequency is significantly higher \ref{fig:pd-eigenvalues}.

\begin{figure}
  \centering
  \includegraphics[width=\columnwidth]{figures/pd-eigenvalues.png}
  \caption{Root locus of the eigenvalues components with respect to speed when
  the gain scheduling in Figure \ref{fig:pd-gains-vs-speed} are applied. Grey
  lines are the uncontrolled bicycle.}
  \label{fig:pd-eigenvalues}
\end{figure}

At very low speeds large steer torques and steer angles are required to
stabilize the system.

\begin{figure}
  \centering
  \includegraphics[width=\columnwidth]{figures/pd-simulation.png}
  \caption{Simulation with saturated steer torque.}
  \label{fig:pd_simulation}
\end{figure}

\section{Full State Feedback}

If you now assume full state feedback and we drive the state to zero as before,
an LQR controller can be realized at each speed. Assumed \(\mathbf{Q}\) and
\(\mathbf{R}\) to be identity for simplicity. I also scale the gains
proportinally to limit the maximum torque that can be applied for some expected
maximum motion. If you dot that, you get some LQR gain scheduling over speed.
The required gains at the uncontrollable points go to infinity, so you
bascially can't stabilize below the largest uncontrollable speed.

\begin{figure}
  \centering
  \includegraphics[width=\columnwidth]{figures/lqr-gains.png}
  \caption{Gain scheduled LQR gains.}
  \label{fig:lqr-gains}
\end{figure}

Using the scheduling in Figure \ref{fig:lqr-gains} give the dynamics in Figure
\ref{fig:lqr-eig}. The LQR solution manages to stabilize the system without
drastically changing the dynamics, in particular the weave frequency is
unchanged. If large gains are permitted, the LQR controller theorectically can
stabilize the bicycle at v=0 speed.

\begin{figure}
  \centering
  \includegraphics[width=\columnwidth]{figures/lqr-eig.png}
  \caption{Root locus of eigenvalue components with the LQR gain scheduling.}
  \label{fig:lqr-eig}
\end{figure}

Here is a low speed simulation with LQR solution.

\begin{figure}
  \centering
  \includegraphics[width=\columnwidth]{figures/lqr-simulation.png}
  \caption{Testing}
  \label{fig:lqr-simulation}
\end{figure}

What do the transfer functions from human steer to roll look like when they are
controlling the closed loop motor controlled bike? For a very low speed, there
is a damped resonant peak around 3 rad/s which changes the way the vehicle
feels in the human control bandwidth.

\begin{figure}
  \centering
  \includegraphics[width=\columnwidth]{figures/lqr-steer-roll-bode-compare-v01.png}
  \caption{Orange is with the motor control on, blue is without.}
  \label{fig:lqr-steer-roll-bode-compare}
\end{figure}

\begin{figure}
  \centering
  \includegraphics[width=\columnwidth]{figures/lqr-steer-roll-bode-compare-v05.png}
  \caption{Orange is with the motor control on, blue is without.}
  \label{fig:lqr-steer-roll-bode-compare}
\end{figure}

Would it be better/possible to ask for the gains that make the Bode plot of
controlled bike match the uncontrolled bike at high speeds? This would give
matching feeling to the rider.

\section{Motor limitations}

Set bounds on steer torque, power, and even bandwith then solve for the open
loop steer torque needed to stablize at various speeds.

Set bounds on steer torque, power, and even bandwith then solve for the open
loop steer torque and solve for the four gains that stabilize the system with
under maximal performance bounds.

Is it of value to add a motor dynamics equation into the mix? A full state
feedback would then require measuring motor current.

\section{Can we mimic other real bicycles?}

Optimization that finds controller parameters for matching dynamics?

\section{Comparison of the gyrobike and the steer motor}

Show how the same dynamics can be produced

Energy costs comparison

\section{Add human control}

What human steer torques are required for normal manuvers? Is it higher for the
stabilized bikes?

Is the stability detrimental to maunerability?

Handling quality metric for the gain scheduled controllers.

\bibliographystyle{plain}
\bibliography{bicycle-steer-control}

\end{document}
