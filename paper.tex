\documentclass[12pt]{article}

\usepackage{amsmath}
\usepackage{graphicx}

\title{Summative Bicycle Steer Control}

\author{Jason K. Moore}
\date{\today}

\begin{document}
\maketitle

\section{Controllability}

There are speed parameter values where the linear Whipple-Carvallo bicycle
model is uncontrollable. Solving the \(det{\mathbf{R}(v)}=0\) returns two
speeds in the range \([0, 10]\) that are uncontrollable.

\begin{figure}
  \centering
  \includegraphics{figures/uncontrolled-eigenvalues-with-rider.png}
  \caption{Real and imaginary parts of the eigenvalues as a function of speed.
  Vertical black lines indicate uncontrollable dynamics.}
  \label{fig:uncontrolled-eigenvalues-with-rider}
\end{figure}

I have no idea why this model is uncontrollable at these speeds.

\section{Stabilization with Roll and Steer Feedback}

One IMU on the handlebars and one on the rear frame easily gives: roll rate and
steer rate. Steer angle sensor gives steer angle. Roll angle must be estimated.

Single motor that applies a torque between the rear frame and the
handlebars/fork.

A full state feedback takes this form:

\begin{align}
  \dot{\bar{x}} = \mathbf{A} \bar{x} + \mathbf{B} \bar{u} \\
  \bar{x} = \begin{bmatrix} \phi \\ \dot{\phi} \\ \delta \\ \dot{\delta} \end{bmatrix} \\
  \bar{u} = \left[ T_\delta \right] \\
  T_\delta = T_{\delta,\textrm{human}} + T_{\delta,\textrm{robot}} \\
  T_{\delta,\textrm{robot}} = k_\phi \phi + k_{\dot{\phi}} \dot{\phi} \\
  \dot{\bar{x}} = \left( \mathbf{A} - \mathbf{B} \mathbf{K} \right) \bar{x} + \mathbf{B} \left[ T_{\delta,\textrm{human}} \right] \\
  \mathbf{K} = \begin{bmatrix} k_\phi & k_{\dot{\phi}} & 0 & 0 \end{bmatrix}
\end{align}

Note that the rider still applies steer torque through the same B matrix and
any roll disturbances also through same B matrix.

\section{Proportional Dervivative Roll Control}

Proportional Derivative Roll Control

\begin{itemize}
  \item What are the achievable closed loop dynamics?
  \item Vary speed and controller gain, show 3D root locus
  \item Pick a speed and vary the controller gain, show 2D root locus
  \item Difference with and without a rigid rider
  \item Could be PD on roll rate criteria (overshoot, decay time) and then pole
    placement with the two parameters for each speed to get similar behavior
    across speeds.
  \item What if we only have rate feedback (raw IMU data)? Does steer rate
    feedback help?
\end{itemize}


Figure: plot the real parts of the eigenvalues vs of A-BK for inf < kphidot <
inf with some intervals of k. We coudl plot strictly the largest real
eigenvalue to minimize clutter on the image.

\begin{figure}
  \centering
  \includegraphics{figures/roll-rate-eig-effect.png}
  \caption{Testing}
  \label{fig:roll-rate-eig-effect}
\end{figure}

Figure: show the gain that produces the lowest stable speed range

\section{Full State Feedback}

- I could frame the whole controller as a full state feedback of essential
bicycle state variables and then show degenerate cases. So [kphi kphidot kdelta
kdeltadot]
- Look at LQR solution for full state feedback and limit the rider steer torque
- What are the achievable closed loop dynamics? (8 gains now)
- Not sufficient to balance at zero speed (what is lowest speed?)
- Show that speed can't affect the rank of the controlability matrix
(analytically)

\begin{figure}
  \centering
  \includegraphics{figures/modal-controllability.png}
  \caption{Testing}
  \label{fig:modal-controllability}
\end{figure}

\begin{figure}
  \centering
  \includegraphics{figures/lqr-eig.png}
  \caption{Testing}
  \label{fig:lqr-eig}
\end{figure}

\section{Effect of Electric Motor Dynamics}

- Limitations in motor power and bandwidth

\section{Can we mimic other real bicycles?}

- Optimization that finds controller parameters for matching dynamics?

\section{Comparison of the gyrobike and the steer motor}

- Show how the same dynamics can be produced
- Energy costs comparison

\section{How does this effect thigns with a leaning rider?}

\section{Add human control}

- Handling quality metric

\section{Time varying controller gains (or as a function of speed)}

\section{Is the stability deteriminal to maunerability?}

\end{document}
